% Intended LaTeX compiler: lualatex
\documentclass[koma,a4paper,utopia,12pt,listings-color,microtype,paralist,colorlinks,urlcolor=red]{org-article}
\usepackage{tikz}
\usetikzlibrary{arrows,decorations.pathmorphing,,backgrounds,positioning,fit,petri,calc,intersections,through,shapes.misc,mindmap,calendar,shadows,mindmap,calendar,graphdrawing,trees,shapes.misc,quotes,angles}
\definecolor{mycolor}{RGB}{139,0,0}
\author{Eason}
\date{\today}
\title{Signals and Systems Chapter 1 Problems Part 3 (1.32-1.47)}
\hypersetup{
 pdfauthor={Eason},
 pdftitle={Signals and Systems Chapter 1 Problems Part 3 (1.32-1.47)},
 pdfkeywords={},
 pdfsubject={This post consists of solutions to problems from 1.21 to 1.31 in chapter 1 of the book Signals and Systems by Oppenheim ,Willsky and Nawab.\cite{oppenheim_signals_1997}},
 pdfcreator={Emacs 26.3 (Org mode 9.2.6)},
 pdflang={English}}
\begin{document}



\section{Problem 1.32}
\label{sec:org919e435}


\begin{prob}[]{prob:1-32}
Let \(x(t)\) be a continuous-time signal, and let
\begin{equation*}
y_{1}(t) = x(2t) \ \mathrm{and} \ y_{2}(t) = x(t/2)
\end{equation*}
The signal \(y_{1}(t)\) represents a speeded up version of \(x(t)\) in the
sense that the duration of the signal is cut in half. Similarly,
\(y_{2}(t)\) represents a slowed down version of \(x(t)\) in the sense that
the duration of the signal is doubled. Consider the following statements:

\begin{enumerate}
\item If \(x(t)\) is periodic, then \(y_{1}(t)\) is periodic.
\item If \(y_{1}(t)\) is periodic, then \(x(t)\) is periodic.
\item If \(x(t)\) is periodic, then \(y_{2}(t)\) is periodic.
\item If \(y_{2}(t)\) is periodic, then \(x(t)\) is periodic.
\end{enumerate}

For each of these statements, determine whether it is true, and if so,
determine the relationship between the fundamental periods of the two
signals considered in the statement. If the statement is not true, produce a
counterexample to it.
\label{prob:1-32}
\end{prob}

\textbf{Problem 1.32a} \label{Problem 1.32a}

Let \(T\) be the fundamental period of \(x(t)\), then we have \(x(t+T) = x(t)\).
So \(y_{1}(t) = x(2t) = x(2t + T) = x(2(t+ \frac{T}{2} ))=
y_{1}(t+\frac{T}{2})\), \(y_{1}(t)\) is periodic and the fundamental period is
\(\frac{T}{2}\) which is consistent with our commonsense that \(y_{1}(t)\) is an
speeded up version of \(x(t)\).

\textbf{Problem 1.32b}

Let \(T\) be the fundamental period of \(y_{1}(t)\), then based on the analysis
of \hyperref[Problem 1.32a]{Problem 1.32a} , we have \(x(t)\) is a slowed down version of \(y_{1}(t)\),
hense \(x(t)\) is periodic and has fundamental period \(2T\).

\textbf{Problem 1.32c} \label{Problem 1.32c}

Let \(T\) be the fundamental period of \(x(t)\), then we have \(x(t+T) = x(t)\).
So \(y_{2}(t) = x(\tfrac{t}{2}) = x(\tfrac{t}{2} + T) = x(\frac{t+ 2T}{2})=
y_{2}(t+2T)\), \(y_{2}(t)\) is periodic and the fundamental period is
\(2T\) which is consistent with our commonsense that \(y_{2}(t)\) is an
slowed down version of \(x(t)\).

\textbf{Problem 1.32d}

Let \(T\) be the fundamental period of \(y_{2}(t)\), then based on the analysis
of \hyperref[Problem 1.32c]{Problem 1.32c} , we have \(x(t)\) is a speeded up version of \(y_{2}(t)\),
hense \(x(t)\) is periodic and has fundamental period \(\frac{T}{2}\).
\section{Problem 1.33}
\label{sec:org7dadf9d}


\begin{prob}[]{prob:1-33}
Let \(x[n]\) be a discrete-time signal, and let
\begin{equation*}
y_{1}[n] = x[2n] \ \mathrm{and}\ y_{2}[n] =
\begin{cases}
x[n/2], & n\ \mathrm{even} \\
0, & n\ \mathrm{odd}
\end{cases}
\end{equation*}
The signals \(y_{1}[n]\) and \(y_{2}[n]\) respectively represent in some
sense the speeded  up and slowed down versions of \(x[n]\). However, it
should be noted that the discrete-time notions of speeded up and slowed down
have subtle differences with respect to their continuous-time counterparts.
Consider the following statements:

\begin{enumerate}
\item If \(x[n]\) is periodic, then \(y_{1}[n]\) is periodic.
\item If \(y_{1}[n]\) is periodic, then \(x[n]\) is periodic.
\item If \(x[n]\) is periodic, then \(y_{2}[n]\) is periodic.
\item If \(y_{2}[n]\) is periodic, then \(x[n]\) is periodic.
\end{enumerate}
\label{prob:1-33}
\end{prob}

\textbf{Problem 1.33a}

Let \(N\) be the fundamental period of \(x[n]\), then \(x[n+N] = x[n]\), and
\(y_{1}[n] = x[2n] = x[2n+N] = y_{1}[n+ \frac{N}{2}]\). If \(N\) is even, then
\(y_{1}[n]\) has fundamental period \(\frac{N}{2}\).  It is a little tricky
that when \(N\) is odd \(\frac{N}{2}\) has no meaning. If \(N\) is odd, we have
to express \(y_{1}[n] = x[2n] = x[2n+2N] = y_{1}[n+ N]\) hence \(y_{1}[n]\) has
fundamental period \(N\) if \(N\) is odd.

\textbf{Problem 1.33b}

We can see \(y_{1}[n]\) is not only a speeded up version but also a down
sampling version of \(x[n]\). we cannot determine whether or not \(y_{1}[n]\)
is periodic based on that \(x[n]\) is periodic. We can only tell that the even
indexed value of \(x[n]\) is period. we know nothing about the odd indexed value
of \(x[n]\). An example is given below:

The given \(x[n]\) is periodic at \(2n\) but is random at \(2n+1\).
\begin{center}
\includegraphics[width=.9\linewidth]{/Users/chaolongzhang/Dropbox/mstemc_hugo/static/img/tikz/figProb1-33b.pdf}
\end{center}


\begin{center}
\includegraphics[width=0.8\textwidth]{/Users/chaolongzhang/Dropbox/mstemc_hugo/static/img/tikz/figProb1-33b.pdf}
\end{center}

\textbf{Problem 1.33c} \label{Problem 1.33c}

\(y_{2}[n]\) can treated as interpolation of \(x[n]\) with zero. So \(y_{2}[n]\)
is periodic with fundamental period \(2N\) where \(N\) is fundamental period of
\(x[n]\).

\textbf{Problem 1.33d}

Based on analysis of \hyperref[Problem 1.33c]{Problem 1.33c} , \(y_{2}[n]\) has fundamental period \(2N\)
where \(N\) is the fundamental period of \(x[n]\).

Based on the fact that only integer is valid for the independent variable of
discrete-time signal, \(y_{2}[n]\) contains all the value of \(x[n]\) at the
even index whereas at the odd index \(y_{2}[n]= 0\).
\section{Problem 1.34}
\label{sec:org20cd069}


\begin{prob}[]{prob:1-34}
In this problem, we explore several of the properties of even and odd
signals.

\begin{enumerate}
\item Show that if \(x[n]\) is an odd signal, then
\begin{equation*}
\sum_{n=-\infty}^{+\infty} x[n] =0
\end{equation*}
\item Show that if \(x_{1}[n]\) is an odd signal and \(x_{2}[n]\) is an even
signal, then \(x_{1}[n]x_{2}[n]\) is an odd signal.
\item Let \(x[n]\) be an arbitrary signal with even and odd parts denoted by
\begin{equation*}
x_{e}[n] = \mathrm{Even}\{ x[n] \}
\end{equation*}
and
\begin{equation*}
x_{o}[n] = \mathrm{Odd} \{ x[n] \}
\end{equation*}
Show that
\begin{equation*}
\sum_{n=-\infty}^{+\infty} x^{2}[n] = \sum_{n=-\infty}^{+\infty} x_{e}^{2}[n] +  \sum_{n=-\infty}^{+\infty} x_{o}^{2}[n]
\end{equation*}
\item Although parts 1-3 have been stated in terms of discrete-time
signals, the analogous properties are also valid in continuous time. To
demonstrate this, show that
\begin{equation*}
\int_{-\infty}^{+\infty}x^{2}(t)dt = \int_{-\infty}^{+\infty} x_{e}^{2}(t)dt + \int_{-\infty}^{+\infty} x_{o}^{2}(t)dt
\end{equation*}
where \(x_{e}(t)\) and \(x_{o}(t)\) are, respectively, the even and odd
parts of \(x(t)\).
\end{enumerate}
\label{prob:1-34}
\end{prob}

\textbf{Problem 1.34a}

If \(x[n]\) is an odd signal, then we have \(x[n] = -x[-n]\) and \(x[0]=0\), so
\begin{eqnarray*}
\sum_{n=-\infty}^{+\infty} x[n] &=& \sum_{n=0}^{+\infty} x[n] + \sum_{-\infty}^{0} x[n] \\
&=& \sum_{n=0}^{\infty} \big( x[n] + x[-n] \big)  \\
&=& 0
\end{eqnarray*}

\textbf{Problem 1.34b}

Because \(x_{1}[n]\) is an odd signal, \(x_{1}[n] = -x_{1}[-n]\). Because
\(x_{2}[n]\) is an even signal, \(x_{2}[n] = x_{2}[-n]\). we have

\begin{equation*}
x_{1}[n]x_{2}[n] = -x_{1}[-n]x_{2}[-n] \\
\end{equation*}

i.e. \(x_{1}[n]x_{2}[n]\) is an odd signal.

\textbf{Problem 1.34c}

\begin{eqnarray*}
x_{e}[n] &=& \frac{x[n] + x[-n]}{2} \\
x_{o}[n] &=& \frac{x[n] - x[-n]}{2}
\end{eqnarray*}
Then we have
\begin{eqnarray*}
\sum_{n=-\infty}^{+\infty} x_{e}^{2}[n]  &=& \sum_{n=-\infty}^{+\infty} \bigg(  \frac{ x[n] + x[-n] }{2} \bigg)^{4}   \\
&=& \sum_{n=-\infty}^{+\infty} \bigg(  \frac{ x^{2}[n]  + x^{2}[-n] + 2x[n]x[-n]}{2} \bigg)
\end{eqnarray*}
\begin{eqnarray*}
\sum_{n=-\infty}^{+\infty} x_{o}^{2}[n]  &=& \sum_{n=-\infty}^{+\infty} \bigg(  \frac{ x[n] - x[-n] }{2} \bigg)^{2}   \\
&=& \sum_{n=-\infty}^{+\infty} \bigg(  \frac{ x^{2}[n]  + x^{2}[-n] - 2x[n]x[-n]}{4} \bigg)
\end{eqnarray*}
Then we add the above two equations and get
\begin{eqnarray*}
\sum_{n=-\infty}^{+\infty} x_{e}^{2}[n] +  \sum_{n=-\infty}^{+\infty} x_{o}^{2}[n]  &=& \sum_{n=-\infty}^{+\infty} \frac{x^{2}[n] + x^{2}[-n]}{2} \\
&=& \sum_{n=-\infty}^{+\infty} x^{2}[n]
\end{eqnarray*}

This conclusion tells us that even we seperate the signal into two parts(even
part and odd part), the sum of energy from the even part and odd part is the
total energy of the original signal.

Let
\begin{eqnarray*}
\int_{-\infty}^{+\infty} x^{2}(t) dt &=& \int_{-\infty}^{+\infty} ( x_{e}(t) + x_{o}(t) )^{2}dt  \\
&=& \int_{-\infty}^{+\infty} x_{e}^{2}(t)dt + \int_{-\infty}^{+\infty} x_{o}^{2}(t) dt + \int_{-\infty}^{+\infty} x_{e}(t)x_{o}(t) dt \\
&=& \int_{-\infty}^{+\infty} x_{e}^{2}(t)dt + \int_{-\infty}^{+\infty} x_{o}^{2}(t) dt
\end{eqnarray*}

Notice that \(x_{e}(t)x_{o}(t)\) is and odd signal, so \(\int_{-\infty}^{+\infty} x_{e}(t)x_{o}(t) dt = 0\)
\section{Problem 1.35}
\label{sec:orgcf4d953}


\begin{prob}[]{prob:1-35}
Consider the periodic discrete-time exponential time signal
\begin{equation*}
x[n] = e^{im(2\pi/N)n}
\end{equation*}
Show that the fundamental period of this signal is
\begin{equation*}
N_{0} = N / \mathrm{gcd}(m,N)
\end{equation*}
where \(\mathrm{gcd}(m,N)\) is the greatest common divisor of \(m\) and
\(N\)----that is, the largest integer that divides both \(m\) and \(N\) an
integral number of times. For example,
\begin{equation*}
\mathrm{gcd}(2,1) = 1, \mathrm{gcd}(2,4) = 2, \mathrm{gcd}(8,12) = 4
\end{equation*}
Note that \(N_{0} = N\) if \(m\) and \(N\) have no factors in common.
\label{prob:1-35}
\end{prob}

\begin{prf}[]{}
Let \(N_{0}\) the fundamental period of \(x[n]\), so that
\begin{equation*}
m(2\pi / N) N_{0} = l\times 2\pi
\end{equation*}
So, we have \(N_{0} = \frac{lN}{m}\) . If \(m\) and \(N\) have no factors in
common, we have \(N_{0} = N\) when \(l = m\). If \(m\) and \(N\) have factors in
common, the fundamental period should be
\begin{equation*}
N_{0} = l \frac{N/ \mathrm{gcd}(m,N) }{m/ \mathrm{gcd}(m,N)}
\end{equation*}

When \(l = m/ \mathrm{gcd}(m,N)\), we have \(N_{0} = \frac{N}{ \mathrm{gcd}(m,N) }\).
\end{prf}
\end{document}
