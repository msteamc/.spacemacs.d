% Intended LaTeX compiler: lualatex
\documentclass[koma,a4paper,utopia,12pt,listings-color,microtype,paralist,colorlinks,urlcolor=red]{org-article}
\usepackage{tikz}
\usetikzlibrary{arrows,decorations.pathmorphing,,backgrounds,positioning,fit,petri,calc,intersections,through,shapes.misc,mindmap,calendar,shadows,mindmap,calendar,graphdrawing,trees}
\author{Eason}
\date{\today}
\title{signals and systems chapter 1 problems}
\hypersetup{
 pdfauthor={Eason},
 pdftitle={signals and systems chapter 1 problems},
 pdfkeywords={},
 pdfsubject={summary of this post},
 pdfcreator={Emacs 26.3 (Org mode 9.2.6)},
 pdflang={English}}
\begin{document}




 \thispagestyle{empty}
 \begingroup
\begin{center}
 \vspace{\baselineskip}
 \textbf{\Huge signal and system chapter 1 problems} \par
 \vspace{2\baselineskip}  \newline
 \textbf{\large Eason Zhang with www.makesteamclear.com \par}
 \vspace{\baselineskip} \newline
  \today \par
 \vspace{\baselineskip}
  \vfill
 \setlength{\unitlength}{3pt}
 \includegraphics[width=0.5\textwidth]{/Users/chaolongzhang/Dropbox/mstemc_hugo/static/img/tikz/eulerEquation.pdf}
 \vfill \vspace{\baselineskip}
 \href{WWW.MAKESTEAMCLEAR.COM}{\Large WWW.MAKESTEAMCLEAR.COM} \par\newline
  makesteamclear is a free project, run by Eason Zhang, to make videos about STEAM in a more approachable way. If you find the contents in this article or the site or the youtube channel helpful, please consider \href{www.makesteamclear.com}{\ensuremath \heartsuit support me\ensuremath\heartsuit }, thanks \par
 \end{center}
 \endgroup

 \newpage \thispagestyle{empty} \textbf{copyright page} \newpage \tableofcontents \newpage
\hspace{0pt}\\

summary of this post

\section{Basic Problems With Answers}
\label{sec:orgf4c241b}



\begin{prob}
Problem 1.1: Express each of the following complex numbers in \emph{Cartesian} form \(x+iy\):
\begin{equation*}
\begin{array}{ccc}
 \frac{1}{2}e^{i\pi}  & \frac{1}{2}e^{-i\pi}  & e^{i\pi/2}  \\
 e^{-i\pi/2}         & e^{i5\pi/2}  & \sqrt{2}e^{i\pi/4}  \\
 \sqrt{2}e^{i9\pi/4} & \sqrt{2}e^{-i9\pi/4}  & \sqrt{2} e^{-i\pi/4}
\end{array}
\end{equation*}
\label{prob11}
\end{prob}

\begin{prob}
Problem 1.2: Express each of the following complex numbers in \emph{polar} form (\(re^{i\theta},
-\pi < \theta \leq \pi\)):
\begin{equation*}
\begin{array}{c{3cm}c{3cm}c{3cm}}
5 & -2 & -3i \\
\frac{1}{2} - i \frac{\sqrt{3}}{2} & 1+i & (1-i)^{2} \\
i(1-i)  & \frac{1+i}{1-i}  & \frac{\sqrt{2} + i \sqrt{2}}{1+i\sqrt{3}}
\end{array}
\end{equation*}
\label{prob12}
\end{prob}

First, let's figure them out using the Euler's formula. For the nine complex
numbers in  \hyperref[prob11]{Problem 1.1}, we have:
\begin{eqnarray*}
\tfrac{1}{2} e^{i\pi } &=&  \tfrac{1}{2} (\cos(\pi) + i\sin(\pi) ) = \tfrac{1}{2} \cos(\pi) = -\tfrac{1}{2} \\
\tfrac{1}{2} e^{-i\pi} &=& \tfrac{1}{2}  (\cos(-\pi) + i\sin(-\pi) ) = \tfrac{1}{2} \cos(-\pi) = -\tfrac{1}{2} \\
e^{i\pi/2} &=& \cos(\pi/2) + i \sin(\pi/2) = i \\
e^{-i\pi/2} &=& \cos(-\pi/2) + i \sin(-\pi/2) = -i \\
e^{i5\pi/2} &=& \cos(5\pi/2) + i \sin(5\pi/2) = \cos(2\pi + \pi/2) + i \sin(2\pi + \pi/2) = i \sin(\pi/2) = i \\
\sqrt{2}e^{i\pi/4} &=& \sqrt{2} ( \cos(\pi/4) + i \sin(\pi/4) ) = \sqrt{2} ( \tfrac{\sqrt{2}}{2} + i\tfrac{\sqrt{2}}{2}  ) = 1 + i \\
 \sqrt{2} e^{i9 \pi/4} &=& \sqrt{2} ( \cos(9\pi/4) + i \sin(9\pi/4) ) = 1 + i \\
 \sqrt{2} e^{-i9 \pi/4} &=& \sqrt{2} ( \cos(-9\pi/4) + i \sin(-9\pi/4) ) = 1 - i \\
 \sqrt{2} e^{-i \pi/4} &=& \sqrt{2} ( \cos(-\pi/4) + i \sin(-\pi/4) ) = 1 - i
\end{eqnarray*}

For the nine complex numbers in  \hyperref[prob12]{Problem 1.2} , we have:

\begin{eqnarray*}
 5 &=& 5 e^{i0} = 5 e^{i(2\pi n)}, n\in \{\ldots, -2,-1,0,1,2,\ldots \} \\
 -2 &=& 2 e^{i\pi} = -2 e^{i0} \\
 -3i &=& 3 e^{-i\pi} \\
 \tfrac{1}{2} - i \tfrac{\sqrt{3}}{2}  &=& e^{-i\pi/3} \\
 1+i &=& \sqrt{2} e^{i\pi/4} \\
 (1-i)^{2} &=& ( \sqrt{2} e^{-i\pi/4} )^{2} = 2 e^{-i\pi/2} \\
 i(1-i) &=& 1 - i = \sqrt{2} e^{-i\pi/4} \\
 \tfrac{1+i}{1-i} &=& \tfrac{ \sqrt{2}e^{i\pi/4} } { \sqrt{2}e^{-i\pi/4} } = e^{i\pi/2} \\
 \tfrac{\sqrt{2} + i \sqrt{2}}{1+i\sqrt{3}} &=& \tfrac{ 2e^{i\pi/4} }{ 2 e^{i\pi/3} } = e^{-i\pi/12}
\end{eqnarray*}


Every complex number \(a+ib\) can be visualized in the complex plane
\(\mathbb{C}\). It can be viewed either as the point with the coordinate
\((a,b)\) or as a vector starting from \(0,0\) to the point \((a,b)\).

Also every complex number \(a+ib\) can be represented in the exponential form
conveniently by using the Euler's formula \(e^{i\alpha} = \cos\alpha +
i\sin\alpha\). One complex number have unique Cartesian form but do not have a
unique expoential form. Taking \(1+i\) for example, its expoential form can be
\(\sqrt{2}e^{i(\pi/4 + 2n\pi)}, n\in \{ \ldots, -2,-1,0,1,2,\ldots \}\). When we
express a complex number in expoential form, it helps to keep a concept of
rotation in mind. In the complex plane \(\mathbb{C}\), a complex number will
return to itself if it rotates a multiple of \(2\pi\) around the origin point
with radius equal to its modulus. please keep the concept of roatation in mind
and it will become increasingly important during our later study.

In the development of complex analysis, \(a+ib\) has another form \(r\angle
\theta\), where \(r\) is the modulus and \(\theta\) the argument (or the angle).
Obviously, this notation is not as good as the expoential form, espacialy when
we want to do complex analysis such as differentiation and integration. We just
mention it here for the sake of completion. The expoential form will be deployed
from now on.


\begin{center}
\includegraphics[width=0.8\textwidth]{/Users/chaolongzhang/Dropbox/mstemc_hugo/static/img/tikz/eulerEquation.pdf}
\end{center}

Now, let's go back to \hyperref[prob11]{Problem 1.1} and Problem \hyperref[prob12]{Problem
1.2}. It's easy to figure the answers out using the Euler's formula. Let's do
more to show them on the complex plane keeping the concept of rotation in mind.

\begin{figure}[htbp]
\centering
\includegraphics[width=0.8\textwidth]{/Users/chaolongzhang/Dropbox/mstemc_hugo/static/img/tikz/problem1_1.pdf}
\caption{\label{problem1-1}
visualized the complex numbers from Problem \hyperref[prob11]{Problem 1.1} in the complex plan}
\end{figure}

From Figure \ref{problem1-1} , taking \(\color{green}{\tfrac{1}{2}e^{i\pi}}\) and
\(\color{red}{\tfrac{1}{2}e^{-i\pi}}\) for example, in the complex plane, they
are the same point \(-\tfrac{1}{2}\) which means \(-\tfrac{1}{2}\) can be
reached by rotating \(\tfrac{1}{2}\) with angle \(\pi\) clockwise or with angle
\(-\pi\) anti-clockwise. Essentially, this is because that \(e^{i\theta} =
e^{i(\theta + 2n\pi)}, n\in \{\ldots,-2,-1,0,1,2,\ldots\}\). It's
straightforward that \(e^{i\pi} = e^{i(\pi + 2(-1)\pi)} = e^{-i\pi}\).

In the end of \hyperref[prob11]{Problem 1.1} and \hyperref[prob11]{Problem 1.1}, I want to say
more about expressing \(\tfrac{1+i}{1-i}\) in its polar form. There are two
methods to get the polar form:
\begin{enumerate}
\item multiply the fraction's numerator and denominator by \(( 1 + i )\)

So we have:
\begin{eqnarray*}
\frac{1+i}{1-i}& = & \frac{(1+i)(1+i)}{(1-i)(1+i)} \\
&=& \frac{2i}{2} = i
\end{eqnarray*}

\item express the numerator and denominator in expoential form first, then do the
following calculation.

\begin{eqnarray*}
\frac{1+i}{1-i} &=& \frac{\sqrt{2}e^{i\pi/4}}{\sqrt{2}e^{-i\pi/4}}  \\
&=& e^{i2\pi/4} = e^{i\pi/2} = i
\end{eqnarray*}
\end{enumerate}
\end{document}
