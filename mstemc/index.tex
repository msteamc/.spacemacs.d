% Intended LaTeX compiler: pdflatex
\documentclass[koma,a4paper,utopia,12pt,listings-color,microtype,paralist,colorlinks,urlcolor=red]{org-article}
       \usepackage{tikz}
       \usetikzlibrary{arrows,decorations.pathmorphing}
       \usetikzlibrary{backgrounds,positioning,fit,petri}
               \usepackage{tikz}
\author{Eason}
\date{\today}
\title{Walk Through the Tutorial 4 of TikZ Manual}
\hypersetup{
 pdfauthor={Eason},
 pdftitle={Walk Through the Tutorial 4 of TikZ Manual},
 pdfkeywords={},
 pdfsubject={Analyze the Elements Example in Chapter 3 of the TikZ manual line by line.},
 pdfcreator={Emacs 26.3 (Org mode 9.2.6)},
 pdflang={English}}
\begin{document}




 \thispagestyle{empty}
 \begingroup
\begin{center}
 \vspace{\baselineskip}
 \textbf{\Huge Walk Through the Tutorial 4 of TikZ Manual} \par
 \vspace{2\baselineskip}  \newline
 \textbf{\large Eason Zhang with www.makesteamclear.com \par}
 \vspace{\baselineskip} \newline
  \today \par
 \vspace{\baselineskip}
  \vfill
 \setlength{\unitlength}{3pt}
 \includegraphics[width=0.5\textwidth]{/Users/chaolongzhang/Dropbox/mstemc_hugo/static/img/tikz/elements.pdf}
 \vfill \vspace{\baselineskip}
 \href{WWW.MAKESTEAMCLEAR.COM}{\Large WWW.MAKESTEAMCLEAR.COM} \par\newline
  makesteamclear is a free project, run by Eason Zhang, to make videos about STEAM in a more approachable way. If you find the contents in this article or the site or the youtube channel helpful, please consider \href{www.makesteamclear.com}{\ensuremath \heartsuit support me\ensuremath\heartsuit }, thanks \par
 \end{center}
 \endgroup

 \newpage \thispagestyle{empty} \textbf{copyright page} \newpage \tableofcontents \newpage
\hspace{0pt}\\

Analyze the Elements Example in Chapter 3 of the TikZ manual line by line.

\lstset{language=[LaTeX]TeX,label=Euclid Amber version of the Elements,caption={Euclid Amber version of the Elements},captionpos=b,firstnumber=1,numbers=left}
\begin{lstlisting}
\begin{tikzpicture}
  [thick,help lines/.style={thin,draw=black!50}]
  \def\A{\textcolor{input}{$A$}}
  \def\B{\textcolor{input}{$B$}}
  \def\C{\textcolor{output}{$C$}}
  \def\D{$D$}
  \def\E{$E$}

  \colorlet{input}{blue!80!black}
  \colorlet{output}{red!70!black}
  \colorlet{triangle}{orange}

  \coordinate [label=left:\A] (A) at ($(0,0) + .0*(rand,rand)$);
  \coordinate [label=right:\B] (B) at ($(1.25,0.25) + .0*(rand,rand)$);

  \draw [input] (A) -- (B);

  \node [name path=D,help lines,draw,label=left:\D] (D) at (A) [circle through = (B)] {};
  \node [name path=E,help lines,draw,label=right:\E] (E) at (B) [circle through = (A)] {};

  \path [name intersections = {of =D and E, by={[label=above:\C]C}}];
  \draw [output] (A) -- (C) -- (B);

  \foreach \point in {A,B,C}
  \fill[black,opacity=.5] (\point) circle (2pt);
  \begin{pgfonlayer}{background}
    \fill[triangle!80] (A) -- (C) -- (B) -- cycle;
  \end{pgfonlayer}

  % \node [below right, text width = 10cm,align = justify] at (4,3) {
  %   \small\textbf{Proposition I} \par
  %   \emph{To construct an \textcolor{triangle}{equilateral triangle}
  %     on a given \textcolor{input}{finite straight line.}}
  %   \par \vskip 1em
  %   Let \A\B\ be the given \textcolor{input}{finite straight line}. \dots
  % };

\end{tikzpicture}
\end{lstlisting}


\begin{center}
\includegraphics[width=0.8\textwidth]{/Users/chaolongzhang/Dropbox/mstemc_hugo/static/img/tikz/elements.pdf}
\end{center}
\end{document}
