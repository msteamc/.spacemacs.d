% Intended LaTeX compiler: xelatex
\documentclass[koma,utopia,letterpaper,captions=tableheading,11pt,listings-sv,microtype,paralist,colorlinks=true,urlcolor=blue]{org-article}
\author{Eason}
\date{\today}
\title{my workflow of creating a video}
\hypersetup{
 pdfauthor={Eason},
 pdftitle={my workflow of creating a video},
 pdfkeywords={},
 pdfsubject={This collection of Videos and Posts describes my workflow of creating a video. Usage of some tools and methods are covered.},
 pdfcreator={Emacs 26.3 (Org mode 9.2.6)},
 pdflang={English}}
\begin{document}

\maketitle
\tableofcontents

During the creation of one video, several tools are tuned to fit my workflow. In
this project several videos and posts will be created to make my workflow clear.

Two targets of this project:
\begin{enumerate}
\item remind me of the configuration of several tools;
\item show somebody else who may be interested in how I work.
\end{enumerate}

I am not ready to detail each tool. If you are interested in any one of them,
please refer to tutorials written by experts or given by the official document.

\section{Emacs}
\label{sec:org232de87}


Emacs lies in the center of my workflow. It behaves like a super-IDE which
assists me finish almost all the steps. Using Emacs, I program and debug, GTD,
write the blogs, export and publish them. Many of the Emacs extensions are
awesome. In particular, Emacs Org is the killer app.

How to config Emacs? After more Than ten years of using Emacs, I choose
spacemacs. Most beginners can start their work using spacemacs with only a
little modifications.

\subsection{program and debug}
\label{sec:orgcb33eca}


I program in python, C/C++, Matlab and some other languages. Emacs acts as a
super-IDE for me.


\subsection{Org mode - journalize your work}
\label{sec:orga18221f}


Org mode definitely need a standalone subsection.

\subsection{Org mode - get things done}
\label{sec:org8f6b45e}



\subsection{Org mode - write a diary}
\label{sec:orgd48d496}

\subsection{Org mode - export your Org file}
\label{sec:org94ce70b}


\subsection{program and debug}
\label{sec:org8c82502}


\subsection{writing latex}
\label{sec:org3f6fba5}
\subsection{org-journal}
\label{sec:org9e0dc38}


I write many notes during a post or a project. Most of my journal is maintained
by \href{https://github.com/bastibe/org-journal}{org-journal: A simple org-mode based journaling mode} . And I set the
\texttt{org-journal-file-type} to \texttt{weekly} So that I have around 52 journal file each
year.

The journals can be searched conveniently. Also they can be integrated into Org
agenda.


\subsection{ox-hugo}
\label{sec:orgeefa353}


\href{https://github.com/kaushalmodi/ox-hugo}{kaushalmodi/ox-hugo: A carefully crafted Org exporter back-end for Hugo}

Because hugo supports markdown at first (it also supports org file later), I use
Emacs Org from the very beginning. I am quite familiar with org-mode, so have no
motivation to dig into markdown. Fortunately, ox-hugo meets all my requirement
and definitely worth a try when you want to stay in Org.

Actually, Emacs have extensions to support markdown quite well. However, using
Org, I can integrate the org file into my agenda.

\section{Hugo}
\label{sec:org4e548ff}


As the fastest framework for building websites, Hugo shocks me by its speed and
flexibility. It is enough to prettify your site using the more than 300+
beautiful themes, from which I am in the mode for Academic theme.

Hugo supports Org file but not so good. Ox-hugo, another extension of Emacs, has
my attention. Serving as a bridge between Emacs and Hugo, ox-hugo helps me
staying in Emacs Org. There is no need for me to write markdown file if Org is
available.

I use ox-hugo as a bridge between Emacs and Hugo. So I do not have to care much
about how to leverage Hugo. Ox-hugo helps me do most of the work.

\section{manim}
\label{sec:org74b3895}



\section{blender}
\label{sec:orge510532}
\end{document}
