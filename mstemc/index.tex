% Intended LaTeX compiler: lualatex
\documentclass[koma,a4paper,utopia,12pt,listings-color,microtype,paralist,colorlinks,urlcolor=red]{org-article}
\usepackage{tikz}
\usetikzlibrary{arrows,decorations.pathmorphing,,backgrounds,positioning,fit,petri,calc,intersections,through,shapes.misc,mindmap,calendar,shadows,mindmap,calendar,graphdrawing,trees}
               \usepackage{tikz}
\author{Eason}
\date{\today}
\title{Walk Through the Tutorial 5 of TikZ Manual}
\hypersetup{
 pdfauthor={Eason},
 pdftitle={Walk Through the Tutorial 5 of TikZ Manual},
 pdfkeywords={},
 pdfsubject={summary of this post},
 pdfcreator={Emacs 26.3 (Org mode 9.2.6)},
 pdflang={English}}
\begin{document}




 \thispagestyle{empty}
 \begingroup
\begin{center}
 \vspace{\baselineskip}
 \textbf{\Huge Walk Through the Tutorial 5 of TikZ Manual} \par
 \vspace{2\baselineskip}  \newline
 \textbf{\large Eason Zhang with www.makesteamclear.com \par}
 \vspace{\baselineskip} \newline
  \today \par
 \vspace{\baselineskip}
  \vfill
 \setlength{\unitlength}{3pt}
 \includegraphics[width=0.5\textwidth]{/Users/chaolongzhang/Dropbox/mstemc_hugo/static/img/tikz/diagram.pdf}
 \vfill \vspace{\baselineskip}
 \href{WWW.MAKESTEAMCLEAR.COM}{\Large WWW.MAKESTEAMCLEAR.COM} \par\newline
  makesteamclear is a free project, run by Eason Zhang, to make videos about STEAM in a more approachable way. If you find the contents in this article or the site or the youtube channel helpful, please consider \href{www.makesteamclear.com}{\ensuremath \heartsuit support me\ensuremath\heartsuit }, thanks \par
 \end{center}
 \endgroup

 \newpage \thispagestyle{empty} \textbf{copyright page} \newpage \tableofcontents \newpage
\hspace{0pt}\\

summary of this post

\begin{center}
\includegraphics[width=0.8\textwidth]{/Users/chaolongzhang/Dropbox/mstemc_hugo/static/img/tikz/diagram.pdf}
\end{center}


This is theorem \ref{theo:1} .
\begin{theo}[Mean Value Theorem]{theo:1}
This is a theorem, and with the
\label{theo:1}
\end{theo}

This is theorem \ref{lem:1} .
\begin{lem}[Mean Value Theorem]{lem:1}
This is a theorem, and with the
\label{lem:1}
\end{lem}


\begin{prf}[]{}
This is a theorem, and with the
\end{prf}

This is problem \ref{prob:1} .
\begin{prob}[]{}
This is a theorem, and with the
\label{prob:1}
\end{prob}

This is problem \ref{solu:1} .
\begin{solu}[]{}
This is a theorem, and with the
\label{solu:1}
\end{solu}
\end{document}
