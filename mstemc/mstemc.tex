% Intended LaTeX compiler: lualatex
\documentclass[koma,a4paper,utopia,12pt,listings-color,microtype,paralist,colorlinks,urlcolor=red]{org-article}
\usepackage{tikz}
\usetikzlibrary{arrows,decorations.pathmorphing,,backgrounds,positioning,fit,petri,calc,intersections,through,shapes.misc,mindmap,calendar,shadows,mindmap,calendar,graphdrawing,trees,shapes.misc,quotes,angles}
\definecolor{mycolor}{RGB}{139,0,0}
\author{Eason}
\date{\today}
\title{Problem 1.21}
\hypersetup{
 pdfauthor={Eason},
 pdftitle={Problem 1.21},
 pdfkeywords={},
 pdfsubject={},
 pdfcreator={Emacs 26.3 (Org mode 9.2.6)},
 pdflang={English}}
\begin{document}




\begin{prob}[]{prob:1-21}
A continuout-time signal \(x(t)\) is shown below. Sketch and label carefully
each of the following signals:
\begin{equation*}
\begin{array}{ll}
\mathbf{(a)}: x(t-1) & \mathbf{(b)}: x(2-t) \\
\mathbf{(c)}: x(2t + 1) & \mathbf{(d)}: x(4-\frac{t}{2}) \\
\mathbf{(e)}: [x(t) + x(-t)] u(t) & \mathbf{(f)}: x(t)\delta(t +\frac{3}{2}) - \delta(t-\frac{3}{2})
\end{array}
\end{equation*}
\begin{center}
\includegraphics[width=0.5\textwidth]{/Users/chaolongzhang/Dropbox/mstemc_hugo/static/img/tikz/figProb1-21-0.pdf}
\end{center}
\label{prob:1-21}
\end{prob}



Before we figure all the sub-problem out. Let's review how \(a\) and \(b\)
affect \(x(at+b)\) . we know that:

\begin{enumerate}
\item when \(|a| < 1\), \(x(t)\) will be linearly stretched;
\item when \(|a|>1\), \(x(t)\) will be linearly compressed;
\item when \(a < 0\), \(x(t)\) will be reversed;
\item when \(b\neq 0\) , \(x(t)\) will be shifted in time.
\item when \(a>0, b>0\), \(x(t)\) will be shifted left.
\item when \(a>0,b<0\), \(x(t)\) will be shifted right.
\item when \(a<0,b>0\), \(x(t)\) will be reversed first then shifted right.
\item when \(a<0,b<0\), \(x(t)\) will be reversed first, then shifted left.
\end{enumerate}

\textbf{Problem 1.21a}

Based on the conclusion given above, we know that \(x(t-1)\) can be obtained by right
shifting \(x(t)\) with step \(1\). So the result can be visualized as follows.


\begin{center}
\includegraphics[width=0.8\textwidth]{/Users/chaolongzhang/Dropbox/mstemc_hugo/static/img/tikz/figProb1-21a.pdf}
\end{center}

\textbf{Problem 1.21b}

For \(x(2-t)\), we re-write it as \(x(-(t-2))\) which can be obtained by first
reversing \(x(t)\) then right shifting it with step \(2\). The result can be
shown as follows

\begin{center}
\includegraphics[width=0.8\textwidth]{/Users/chaolongzhang/Dropbox/mstemc_hugo/static/img/tikz/figProblem1-21b.pdf}
\end{center}

\textbf{Problem 1.21c}

\(x(2t + 1)\) can be re-written as \(x(2(t+\frac{1}{2}))\) which means that
\(x(2t+1)\) can be obtained by first linearly compressing \(x(t)\) with factor 2
then left shifting compressed signal by \(\frac{1}{2}\).

\begin{center}
\includegraphics[width=0.8\textwidth]{/Users/chaolongzhang/Dropbox/mstemc_hugo/static/img/tikz/figProb1-21c.pdf}
\end{center}

\textbf{Problem 1.21d}

\(x(4-\frac{t}{2})\) can be re-written as \(x( -\frac{1}{2}( t - 8 ) )\) which
means that \(x(4-\frac{t}{2})\)  can be obtained by first linearly stretching
\(x(t)\) by factor \(2\) then right shifting it with step \(8\).

\begin{center}
\includegraphics[width=0.8\textwidth]{/Users/chaolongzhang/Dropbox/mstemc_hugo/static/img/tikz/figProb1-21d.pdf}
\end{center}

\textbf{Problem 1.21e}

At first glance, \(\big[x(t) + x(-t)\big] u(t)\) is \(0\) if \(t<0\) .

If \(t > 0\), we have to figure out \(x( -t )\). After obtaining \(x(-t)\), we
add the right part of \(x(t)\) and left part of \(x(-t)\) then we have \([x(t) +
x(-t)]u(t)\)

\begin{center}
\includegraphics[width=0.8\textwidth]{/Users/chaolongzhang/Dropbox/mstemc_hugo/static/img/tikz/figProblem1-21e.pdf}
\end{center}

\(x(t)\delta(t +\frac{3}{2}) - \delta(t-\frac{3}{2})\) only have two points with
non-negative values: \(t=\frac{3}{2}\) and \(t=-\frac{3}{2}\).

\begin{center}
\includegraphics[width=0.8\textwidth]{/Users/chaolongzhang/Dropbox/mstemc_hugo/static/img/tikz/figProb1-21f.pdf}
\end{center}
\end{document}
