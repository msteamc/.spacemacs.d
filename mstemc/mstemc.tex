% Intended LaTeX compiler: lualatex
\documentclass[koma,a4paper,utopia,12pt,listings-color,microtype,paralist,colorlinks,urlcolor=red]{org-article}
\usepackage{tikz}
\usetikzlibrary{arrows,decorations.pathmorphing,,backgrounds,positioning,fit,petri,calc,intersections,through,shapes.misc,mindmap,calendar,shadows,mindmap,calendar,graphdrawing,trees}
\author{Eason}
\date{\today}
\title{Problem 1.4}
\hypersetup{
 pdfauthor={Eason},
 pdftitle={Problem 1.4},
 pdfkeywords={},
 pdfsubject={},
 pdfcreator={Emacs 26.3 (Org mode 9.2.6)},
 pdflang={English}}
\begin{document}



\begin{prob}[]{prob:1-4}
 Problem 1.4 Let \(x[n]\) be a signal with \(x[n]=0\) for\(n< -2\) and \(n>4\),
for each signal given below, determine the values of \(n\) for which it is
guaranteed to be zero.
\begin{eqnarray*}
\mathbf{ (a) } & & x[n-3] \\
\mathbf{ (b) } & & x[n+4] \\
\mathbf{ (c) } & & x[-n]  \\
\mathbf{ (d) } & & x[-n+2] \\
\mathbf{ (e) } & & x[-n-2]
\end{eqnarray*}
\label{prob:1-4}
\end{prob}

For the given signals \(\mathbf{(a)}\) to \(\mathbf{(e)}\), the transformations
of the variable \(n\) will change the interval in which the signals are zero.
For the convenience of calculation, we write the origin signal as:
\begin{equation*}
x[m] = 0, m< -2 \quad \mathrm{and} \quad m>4
\end{equation*}

We can visualize \(x(m)\) as below (I just give an example, you can name any
signal that satisfy \(x[m] = 0, m< -2 \ \mathrm{and} \ m>4\)):

\begin{center}
\includegraphics[width=0.5\textwidth]{/Users/chaolongzhang/Dropbox/mstemc_hugo/static/img/tikz/figProb1-4.pdf}
\end{center}


\(\mathbf{(a)}:x[n-3]\)

For signal \(\mathbf{(a)}\), to get the interval where \(x[n-3] = 0\), we have:
\begin{eqnarray*}
  t=n-3 &<& -2 \\
  t=n-3 &>& 4
\end{eqnarray*}
 Then, we have \(n < 1 \ \mathrm{and} \ n > 7\) from which we can
see that the new signal is a right shift with step three relative to the origin
signal. The new signal is delayed with three.


\begin{center}
\includegraphics[width=0.5\textwidth]{/Users/chaolongzhang/Dropbox/mstemc_hugo/static/img/tikz/figProb1-4a.pdf}
\end{center}

\(\mathbf{(b)}:x[n+4]\)

For signal \(\mathbf{(b)}\), we have:
\begin{eqnarray*}
t=n+4&<&-2 \\
t=n+4&>&4
\end{eqnarray*}
Then, we have \(n<-6\ \mathrm{and}\ n>0\) from which we can see that the new
signal is a left shift with step four relative to the origin singal. The new
signal is advanced with four.

\begin{center}
\includegraphics[width=0.5\textwidth]{/Users/chaolongzhang/Dropbox/mstemc_hugo/static/img/tikz/figProb1-4b.pdf}
\end{center}


\(\mathbf{(c)}: x[-n]\)

For signal \(\mathbf{( c )}\), we have:
\begin{eqnarray*}
t=-n&<&-2 \\
t=-n&>&4
\end{eqnarray*}

Then, we have \(n>2\ \mathrm{and} \ n<-4\) from which we can see that the new
signal is a reversal of the origin signal.

\begin{center}
\includegraphics[width=0.5\textwidth]{/Users/chaolongzhang/Dropbox/mstemc_hugo/static/img/tikz/figProb1-4c.pdf}
\end{center}


\(\mathbf{(d)}: x[-n+2]\)

for signal \(\mathbf{(d)}\), we have:
\begin{eqnarray*}
t=-n+2&<&-2 \\
t=-n+2&>&4
\end{eqnarray*}
Then, we have \(n>4\ \mathrm{and}\ n<-2\). For \(x[-n+2]\), we can first
flip the original signal then right shift the flipped signal by 2. Notice the
contents in the brackets \(-(n+2)\). I would like treat it as \(-(n-2)\), by
which I know that the minus symbol means reversal and \(-2\) means right
shift by 2.

\begin{center}
\includegraphics[width=0.5\textwidth]{/Users/chaolongzhang/Dropbox/mstemc_hugo/static/img/tikz/figProb1-4d.pdf}
\end{center}



\(\mathbf{(e)}: x[-n-2]\)

For signal \(\mathbf{(e)}\), we have:

\begin{eqnarray*}
t=-n-2&<&-2 \\
t=-n-2&>&4
\end{eqnarray*}
Then, we have \(n>0 \ \mathrm{and}\ n<-6\). To get the new signal, we have
to filp the original signal first then left shift the flipped one by two.

\begin{center}
\includegraphics[width=0.5\textwidth]{/Users/chaolongzhang/Dropbox/mstemc_hugo/static/img/tikz/figProb1-4e.pdf}
\end{center}
\end{document}
