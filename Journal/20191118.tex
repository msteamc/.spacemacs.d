% Intended LaTeX compiler: xelatex
\documentclass[koma,utopia,letterpaper,captions=tableheading,11pt,listings-sv,microtype,paralist,colorlinks=true,urlcolor=blue]{org-article}
\author{Eason}
\date{\today}
\title{20191119}
\hypersetup{
 pdfauthor={Eason},
 pdftitle={20191119},
 pdfkeywords={},
 pdfsubject={},
 pdfcreator={Emacs 26.3 (Org mode 9.2.6)},
 pdflang={English}}
\begin{document}

\maketitle
\tableofcontents

\section{Tuesday, 11/19/19}
\label{sec:orgaf7c73e}
\subsection{21:55 this is a new week}
\label{sec:org784f7a1}
STEAM: science, technology, engineering, art and math

Today, I listen to an talk between 3blue1brown and numberphile.

\section{Wednesday, 11/20/19}
\label{sec:org7da6c22}

\section{Sunday, 11/24/19}
\label{sec:org9fb7c83}
\subsection{introduction to signal and system}
\label{sec:org4132682}
\subsubsection{preface}
\label{sec:org30a21fe}


While students studying signals and systems should certainly have a solid
foundation in disciplines based on the laws of physics, they must also have a
solid foundation in disciplines based on the laws of physics, they must also
have a firm grounding in the use of computers for the analysis of phenomena
and the implementation of systems and algorithms. As a consequence, engineering
curricula now reflect a blend of subjects, some involving continuous-time models
and others focusing on the use of computers and discrete representations. For
these reasons, signals and systems courses that bring discrete-time and
continuous-time concepts together in a unified way play an increasingly
important role in the education of engineering students and in their preparation
for current and future developments in their chosen fields.

It has pedagogical advantage by taking both the continuous-time and
discrete-time in mind.

Chapter 1 introduces block diagram representations of interconnections of
systems and discusses several basic system properties such as causality,
linearity and time-invariance.

The concepts of signals and systems arise in a wide variety of fields, and the
ideas and techniques associated with these concepts play an important role in
such diverse areas of science and technology such communications, aeronautics
and astronautics, circuit design, acoustics, seismology, biomedical engineering,
energy generation and distribution systems, chemical process control, and speech
processing. Because of its importance, it worth to take time to digest each
single idea and concept appearing in such course. Also, it will be greatly
helpful for the newbie to have an open course to help them dive into the world
of signals and systems. I tried my best to give the best explanation of the
concept in this course and comprehensive details are given during my creation of
videos and posts. It takes time to produce one piece of work so that sometimes I
think it was a grind. However, for the sake of beautiful result, I pretty enjoy
the process of doing such creation.

The signals, which are functions of one or more independent variables, contains
information about the behavior or nature of some phenomenon, whereas the systems
respond to particular signals by producing other signals or some desired
behavior.

Examples:
\begin{enumerate}
\item Voltages and currents as a function of time in an electrical circuit are
examples of signals. and a circuit is itself an example of a system, which in
this case responds to applied voltages and currents.
\item As another example, when an automobile driver depresses the accelerator
pedal, the automobile responds by increasing the speed of the vehicle.
\item A computer program for the automated diagnosis of electrocardiograms can be
viewed as a system which has as its input a digitized electrocardiogram and
which produces estimates of parameters such as heart rate as outputs.
\item A camera is a system that receives light from different sources and reflected
from objects and produces a photograph.
\item A robot arm is a system whose movements are the response to control inputs.
\end{enumerate}


The purpose of signals ans systems:
\begin{enumerate}
\item characterizing the system in detail to understand how it will respond to
various inputs.
\item design systems to process signals in particular ways. such enhancement and
restoration of certain signals, extraction of specific pieces of information
from signals.
\item modify or control the characteristics of a given system.
\end{enumerate}


\subsubsection{15:39 Basics of signals and systems}
\label{sec:orgf198687}


This section is from \href{http://www.di.univr.it/documenti/OccorrenzaIns/matdid/matdid744681.pdf}{Signals-and-Systems.ppt}  by Gloria Menegaz. The textbook
is : \emph{Signal Processing and Linear Systems} by B.P. Lathi, CRC Press.

\subsubsection{definition of signals and systems}
\label{sec:org9f44052}


A signal is defined as any physical quantity that varies with time, space, or
any other independent variable or variables. Mathematically, we describe a
signal as a function of one or more independent variables. Signals convey
information.

A continuous-time signal is a quantity of interest that
depends on an independent variable, where we usually think of the independent
variable as time. In the real world, physical quantities take on real numerical
values, though it turns out that sometimes it is mathematically convenient to
consider \emph{complex-valued functions of t}. However, the default is real-valued
\(x(t)\), and indeed the type of sketch exhibited about is valid only for
real-valued signals.

Remarks:
\begin{enumerate}
\item A continuous-time signal is not necessarily a continuous function as defined
in calculus. They are two different definition.
\item The independent variable need not to be time, it could has other meaning. But
considering it as time does not affect we study the concepts of signals and
systems.
\end{enumerate}

A discrete-time signal is a sequence of values of interest, where the integer
index can be thought of as a time index, and the values in the sequence
represent some physical quantity of interest. Because many discrete-time signals
arise as equally-spaced samples of a continuous-time signal, it is often more
convenient to think of the index as the "sample number".

Systems process signals to produce a modified or transformed version of the
original signal. For the information and communication technology, signals and
systems helps engineer to analyze and design the signals and systems they meet
in daily life.






\subsubsection{summary of chapter 1}
\label{sec:orgde426c3}


We developed a number of basic concepts related to continuous-time and
discrete-time signals and systems. We have presented both an intuitive picture
of what signals and systems are through several examples and a mathematical
representation for signals and systems that we will use throughout the book.
Specifically, we introduced graphical and mathematical representation of signals
and used these representations in performing transformations of the independent
variable. We also defined and examined several basic signals, both in continuous
time and in discrete time. These included complex exponential signals,
sinusoidal signals, and unit impulse and step functions. In addition, we
investigated the concept of periodicity for continuous-time and discrete-time
signals.

In developing some of the elementary ideas related to systems, we introduced
block diagrams to facilitate our discussions concerning the interconnection of
systems, and we defined a number of important properties of systems, including
causality, stability, time-invariance, and linearity.

The primary focus in most of this book will be on the class of LTI systems, both
in continuous time and discrete time. These systems play a particularly
important role in system analysis and design, in part due to the fact that many
systems encountered in nature can be successfully modeled as linear and time
invariant. Furthermore, as we shall see in the following chapters, the
properties of linearity and time invariance allows us to analyze in detail the
behavior of LTI systems.

\subsubsection{references}
\label{sec:orgbf8c476}


Other references are:
\begin{enumerate}
\item Signals and systems by Richard Baraniuk's lecture
notes.\href{https://web.itu.edu.tr/hulyayalcin/Signal\_Processing\_Books/2003\_Richard\_Baraniuk\_Signals\_and\_Systems.pdf}{2003\_Richard\_Baraniuk\_Signals\_and\_Systems.pdf}
\item Digital Signal Processing by John G. Proakis, Dimitris K Manolakis\href{https://engineering.purdue.edu/\~ee538/DSP\_Text\_3rdEdition.pdf}{Digital Signal Processing: Principles, Algorithms \& Applications}

\item Signal processing and linear systems, Schaun's outline of digital signal
processing.
\item Foundations of Signal Processing.\href{http://fourierandwavelets.org/}{Foundations of Signal Processing and Fourier
and Wavelet Signal Processing}

Together with Fourier and Wavelet Signal Processing (to be published by CUP),
the two books aim to present the essential principles in signal processing
along with mathematical tools and algorithms for signal representation. They
comprehensively cover both classical Fourier techniques and newer basis
constructions from filter banks and multiresolution analysis—wavelets.
Furthermore, they gives a synthetic view from basic mathematical principles,
to construction of bases, all the way to concrete applications.

\item Signal processing for communications, by Palo Prandoni and Martin
Vetterli. \href{https://www.sp4comm.org/index.html}{Signal Processing for Communications}

With a novel, less formal approach to the subject, the authors have written a
book with the conviction that signal processing should be taught to be fun.
The treatment is less focused on the mathematics and more on the conceptual
and practical aspects but the book remains an engineering text, with the goal
of helping students solve real-world problems. In this vein, the last chapter
pulls together all the topics discussed throughout the book into an in-depth
look at the development of an end-to-end communication system, namely, a
modem for communicating digital information over an analog channel.

\item Lecture notes of ELE 301 by \href{https://www.princeton.edu/\~cuff/}{Paul W. Cuff} . The\href{https://www.princeton.edu/\~cuff/ele301/files/lecture1\_2.pdf}{Lecture 1 ELE 301: Signals and
Systems}  process some basic of the signals and systems

\item Signal and system analysis by Jianping Yao\href{http://www.site.uottawa.ca/\~jpyao/courses/E3125B\_Fall\_2019.html}{ELG3120} .

\item Notes for signals and Systems by Wilson J. Rugh\href{https://pages.jh.edu/\~bcooper8/sigma\_files/courses/214/signalsandsystemsnotes.pdf}{Notes for Signals and Systems}
. And in the site\href{https://pages.jh.edu/\~signals/}{Signals, Systems, and Control Demonstrations} has some other
demos about signals and systems.

\item \href{https://ptolemy.berkeley.edu/books/leevaraiya/}{Lee and Varaiya, Structure and Interpretation of Signals and Systems}

Signals convey information. Systems transform signals. This book introduces
the mathematical models used to design and understand both. It is intended
for students interested in developing a deep understanding of how to
digitally create and manipulate signals to measure and control the physical
world and to enhance human experience and communication. This book is based
on several years of successful classroom use at the University of California,
Berkeley. The material starts with an early introduction to applications,
well before students have built up enough theory to fully analyze the
applications. This motivates students to learn the theory and allows students
to master signals and systems at the sophomore level. The material motivates
signals and systems through sound and images. Calculus is the only
prerequisite.

 And the PDF version of the book is \href{https://ptolemy.berkeley.edu/books/leevaraiya/releases/LeeVaraiya\_DigitalV2\_04.pdf}{here}. There is also \href{https://ptolemy.berkeley.edu/books/leevaraiya/releases/LabManualV1\_AddisonWesley.pdf}{an accompanied
laboratory} manual using matlab.

\item Gurdal Arslan has several \href{http://www2.hawaii.edu/\~gurdal/index\_files/Page386.htm}{Courses} about signals and systems.

\item \href{http://www.eas.uccs.edu/\~mwickert/ece2610/}{ECE2610 Introduction to Signals and Systems}

Mathematical representation of signals and systems; spectrum representation;
representation of signals by sample values; discrete-time filter
characterization and response; the z-transform; continuous-time signals and
linear, time-invariant systems; frequency response; continuous-time Fourier
transform and application to signals and systems. Include lectures,
demonstrations, and laboratory assignments.

\item Dr Wickert's website \href{http://www.eas.uccs.edu/\~mwickert/}{Dr. Wickert's Info Center}  has quite a lot resources
for a engineering student.
\end{enumerate}
\end{document}
